% Copyright ©2025  
% THE AIMS-SENEGAL ACADEMIC TEAM 	
% version 3.0 (redesigned edition, August 2025)

\documentclass[12pt,aspectratio=169]{beamer}
\usetheme{Madrid}
\usecolortheme{whale}
% AIMS official colors - modernized palette
\definecolor{aimscolour}{rgb}{0.54,0.11,0.01}
\definecolor{modernblue}{rgb}{0.2,0.4,0.8}
\definecolor{softgray}{rgb}{0.95,0.95,0.95}
\definecolor{accentorange}{rgb}{1.0,0.6,0.2}

% Use modern blue as primary color
\usecolortheme[named=modernblue]{structure}

% Enhanced packages for modern design
\usepackage[utf8]{inputenc}
\usepackage{amsmath}
\usepackage{csquotes}
\usepackage{amsfonts}
\usepackage{amssymb}
\usepackage{setspace}
\usepackage{gensymb}
\usepackage{graphicx}
\usepackage[section]{placeins}
\usepackage[linesnumbered,ruled,vlined]{algorithm2e}
\usepackage{subcaption}
\usepackage{tikz}
\usepackage{fontawesome5}  % Modern icons
\usepackage{xcolor}

% Modern spacing
\onehalfspacing

% Subtle background without logo overlay
\setbeamertemplate{background}{}

% Modern block styling with transparency
\definecolor{UMRed}{rgb}{0.5, 0.09, 0.09}
\addtobeamertemplate{block begin}{\pgfsetfillopacity{0.8}}{\pgfsetfillopacity{1}}
\addtobeamertemplate{block alerted begin}{\pgfsetfillopacity{0.8}}{\pgfsetfillopacity{1}}
\addtobeamertemplate{block example begin}{\pgfsetfillopacity{0.8}}{\pgfsetfillopacity{1}}

% Enhanced proof environment
\addtobeamertemplate{proof begin}{%
    \setbeamercolor{block title}{fg=white,bg=modernblue}%
    \setbeamercolor{block body}{fg=black, bg=blue!10!white}%
}{}

% Custom theorem environments
\newtheorem*{proposition}{Proposition}
\newtheorem*{remark}{Remark}

% Bibliography styling
\usepackage[style=phys]{biblatex}
\addbibresource{bibliography.bib}

% Modern section transitions
\AtBeginSubsection[]
{
   \begin{frame}
        \frametitle{\faList \, Section Overview}
        \tableofcontents[currentsection,currentsubsection]
   \end{frame}
}

% Equation numbering
\counterwithin{equation}{section}

% Advanced differential notation
\usepackage[roman, thin, thinp, thinc]{esdiff}
\usepackage{bm}

% Modern footer with page numbers
\setbeamertemplate{page number in head/foot}[totalframenumber]
\beamertemplatenavigationsymbolsempty
\setbeamerfont{page number in head/foot}{size=\footnotesize}

% Algorithm styling
\SetKwInput{KwInput}{Input}
\SetKwInput{KwOutput}{Output}

% Enhanced comment styling
\newcommand\mycommfont[1]{\footnotesize\ttfamily\textcolor{modernblue}{#1}}
\SetCommentSty{mycommfont}

% Modern frame title formatting
\setbeamerfont{frametitle}{size=\Large,series=\bfseries}
\setbeamerfont{framesubtitle}{size=\normalsize,series=\normalfont}

% Enhanced itemize symbols
\setbeamertemplate{itemize item}{\faChevronRight}
\setbeamertemplate{itemize subitem}{\faAngleRight}
\setbeamertemplate{itemize subsubitem}{\faCircle}

% Modern alert and example block colors
\setbeamercolor{block title alerted}{fg=white,bg=red!70!black}
\setbeamercolor{block body alerted}{fg=black,bg=red!10!white}
\setbeamercolor{block title example}{fg=white,bg=green!70!black}
\setbeamercolor{block body example}{fg=black,bg=green!10!white}

\usepackage{tikz-3dplot}
\usetikzlibrary{angles, quotes, decorations.pathmorphing, shadows, shapes.geometric, positioning}
\usepackage{subcaption}  % for subfigure environments
\usepackage{fontawesome5}  % for modern icons

% Title and Author Information
\author[AMMI Tutors]{
    \textbf{AMMI Teaching Team} \\
    \small African Masters of Machine Intelligence \\
    \small BootCamp 2025
}

\title[Linear Algebra Fundamentals]{
    \textbf{Linear Algebra Fundamentals} \\
    \large Vectors, Matrices, Decomposition \& Tensors \\
    \vspace{0.5cm}
    \small \faGraduationCap \, Essential Mathematics for Machine Learning
}

\institute[AIMS-Senegal]{
    \large African Institute for Mathematical Sciences \\
    \small Senegal
}

\date{\today}

% Remove default navigation symbols and configure appearance
\setbeamertemplate{navigation symbols}{}
\setbeamercovered{transparent}
\setbeamertemplate{page number in head/foot}[totalframenumber]

\begin{document}
\tdplotsetmaincoords{60}{110}

% Custom title slide
\begin{frame}[plain]
\begin{tikzpicture}[overlay, remember picture]
    % Background gradient
    \fill[blue!5!white] (current page.south west) rectangle (current page.north east);
    
    % Decorative geometric shapes
    \node[circle, fill=blue!20, opacity=0.3, minimum size=3cm] at ([xshift=5cm,yshift=2cm]current page.north west) {};
    \node[circle, fill=red!20, opacity=0.3, minimum size=2cm] at ([xshift=-3cm,yshift=-3cm]current page.south east) {};
    
    % Title content
    \node[anchor=center] at (current page.center) {
        \begin{minipage}{0.8\textwidth}
            \centering
            {\huge \textbf{Linear Algebra Fundamentals}} \\[0.5cm]
            {\Large Vectors, Matrices, Decomposition \& Tensors} \\[0.8cm]
            {\large \faGraduationCap \, Essential Mathematics for Machine Learning} \\[1.5cm]
            
            {\large \textbf{AMMI Teaching Team}} \\[0.3cm]
            {\normalsize African Masters of Machine Intelligence} \\
            {\normalsize BootCamp 2025} \\[1cm]
            
            {\normalsize African Institute for Mathematical Sciences, Senegal} \\[0.5cm]
            {\small \today}
        \end{minipage}
    };
\end{tikzpicture}
\end{frame}

% Table of Contents with modern styling
\begin{frame}[allowframebreaks]
\frametitle{\faList \, Course Overview}
\begin{columns}[T]
    \begin{column}{0.7\textwidth}
        \tableofcontents
    \end{column}
    \begin{column}{0.25\textwidth}
        \begin{tikzpicture}
            \node[draw, rounded corners, fill=blue!20, text width=3cm, align=center] {
                \small \textbf{Learning Goals} \\[0.2cm]
                \faCheckCircle \, Master vector operations \\[0.1cm]
                \faCheckCircle \, Understand matrices \\[0.1cm]
                \faCheckCircle \, Apply to ML
            };
        \end{tikzpicture}
    \end{column}
\end{columns}
\end{frame}

\section[Motivation]{\faRocket \, Motivation}
\subsection{Why Linear Algebra Matters in AI/ML}

\begin{frame}{\faRocket \, Why Linear Algebra is Essential for AI/ML}
\framesubtitle{The mathematical foundation of modern artificial intelligence}

\begin{columns}[T]
    \begin{column}{0.6\textwidth}
        \begin{itemize}[<+->]
            \item \faDatabase \, \textbf{Data Representation:} \\
                  \small Vectors and matrices encode all data types
            \item \faCogs \, \textbf{Algorithm Core:} \\
                  \small Powers neural networks, transformers, etc.
            \item \faChartLine \, \textbf{Optimization:} \\
                  \small Gradient descent relies on matrix calculus
            \item \faCompress \, \textbf{Dimensionality Reduction:} \\
                  \small PCA, SVD enable efficient computation
        \end{itemize}
    \end{column}
    \begin{column}{0.35\textwidth}
        \begin{tikzpicture}[scale=0.8]
            % Neural network visualization
            \foreach \y in {1,2,3} {
                \node[circle, draw, fill=blue!20] (i\y) at (0,\y) {};
            }
            \foreach \y in {1,2,3,4} {
                \node[circle, draw, fill=green!20] (h\y) at (2,\y-0.5) {};
            }
            \foreach \y in {1,2} {
                \node[circle, draw, fill=red!20] (o\y) at (4,\y+0.5) {};
            }
            
            % Connections
            \foreach \i in {1,2,3} {
                \foreach \j in {1,2,3,4} {
                    \draw[->] (i\i) -- (h\j);
                }
            }
            \foreach \i in {1,2,3,4} {
                \foreach \j in {1,2} {
                    \draw[->] (h\i) -- (o\j);
                }
            }
            
            \node[below] at (2,-0.5) {\small Matrix Operations};
        \end{tikzpicture}
    \end{column}
\end{columns}
\end{frame}

\begin{frame}{\faLightbulb \, Real-World Applications}
\framesubtitle{Where linear algebra powers modern technology}

\begin{columns}[T]
    \begin{column}{0.45\textwidth}
        \begin{block}{\faRobot \, Deep Learning}
            \begin{itemize}
                \item Matrix multiplications in layers
                \item Backpropagation algorithms
                \item Tensor operations
            \end{itemize}
        \end{block}
        
        \begin{block}{\faSearch \, Computer Vision}
            \begin{itemize}
                \item Image transformations
                \item Convolutional operations
                \item Feature extraction
            \end{itemize}
        \end{block}
    \end{column}
    
    \begin{column}{0.45\textwidth}
        \begin{block}{\faComments \, Natural Language}
            \begin{itemize}
                \item Word embeddings
                \item Attention mechanisms
                \item Transformer architectures
            \end{itemize}
        \end{block}
        
        \begin{block}{\faUsers \, Recommendations}
            \begin{itemize}
                \item Matrix factorization
                \item Collaborative filtering
                \item Graph-based learning
            \end{itemize}
        \end{block}
    \end{column}
\end{columns}

\vspace{0.5cm}
\centering
\textit{"Linear algebra is the language of data science and AI"} - Gilbert Strang
\end{frame}

\section[Vectors]{\faArrowRight \, Vectors}
\subsection{Definitions and Fundamental Concepts}

\begin{frame}{\faArrowRight \, Vectors: Definition and Notation}
\framesubtitle{The building blocks of linear algebra}

\begin{columns}[T]
    \begin{column}{0.55\textwidth}
        \begin{definition}[Vector]
            A \textbf{vector} is an ordered list of numbers.
        \end{definition}
        
        \vspace{0.3cm}
        \textbf{Representations:}
        \begin{itemize}
            \item \textbf{Row form:} 
            \begin{align}
                \mathbf{x} = \left[ x_1, x_2, \ldots, x_n \right] \label{eq:row}
            \end{align}
            
            \item \textbf{Column form:} 
            \begin{align}
                \mathbf{x} = \begin{bmatrix}
                    x_1 \\
                    x_2 \\
                    \vdots \\
                    x_n
                \end{bmatrix} \label{eq:column}
            \end{align}
        \end{itemize}
    \end{column}
    
    \begin{column}{0.4\textwidth}
        \begin{tikzpicture}[scale=0.8]
            % Vector visualization
            \draw[->] (0,0) -- (3,0) node[below] {$x_1$};
            \draw[->] (0,0) -- (0,3) node[left] {$x_2$};
            \draw[->, thick, blue] (0,0) -- (2.5,2) node[above right] {$\mathbf{x}$};
            
            % Components
            \draw[dashed, gray] (2.5,2) -- (2.5,0) node[below] {$x_1$};
            \draw[dashed, gray] (2.5,2) -- (0,2) node[left] {$x_2$};
            
            \node[below right] at (1.2,1) {\small 2D Vector};
        \end{tikzpicture}
        
        \vspace{0.5cm}
        \begin{alertblock}{Key Point}
            \small Vectors represent both \textbf{magnitude} and \textbf{direction}
        \end{alertblock}
    \end{column}
\end{columns}
\end{frame}

\begin{frame}{\faInfoCircle \, Vector Properties and Terminology}
\framesubtitle{Essential definitions and notation}

\begin{columns}[T]
    \begin{column}{0.5\textwidth}
        \begin{itemize}[<+->]
            \item \textbf{Components:} The numbers $x_1, x_2, \ldots, x_n$ are called \emph{elements}, \emph{entries}, or \emph{components}
            
            \item \textbf{Dimension:} The number of elements determines the vector's \emph{dimension} or \emph{length}
            
            \item \textbf{Notation:} 
            \begin{itemize}
                \item Lowercase letters: $\mathbf{a}, \mathbf{x}, \mathbf{p}, \mathbf{r}$
                \item Sometimes: $\vec{x}$ or $\boldsymbol{x}$
                \item $i$-th element: $x_i$ or $[\mathbf{x}]_i$
            \end{itemize}
        \end{itemize}
    \end{column}
    
    \begin{column}{0.45\textwidth}
        \begin{example}[Concrete Examples]
            \begin{align}
                \mathbf{a} &= \begin{bmatrix} 1 \\ 2 \\ -1 \end{bmatrix} \quad \text{(3-vector)}\\[0.5cm]
                \mathbf{b} &= \begin{bmatrix} -2 \\ 1.1 \\ 0 \\ 5 \end{bmatrix} \quad \text{(4-vector)}
            \end{align}
        \end{example}
        
        \vspace{0.3cm}
        \begin{block}{Equality}
            $\mathbf{a} = \mathbf{b}$ if and only if $a_i = b_i$ for all $i$
        \end{block}
    \end{column}
\end{columns}
\end{frame}

\begin{frame}{\faLayerGroup \, Special Types of Vectors}
\framesubtitle{Important vector categories in linear algebra}

\begin{columns}[T]
    \begin{column}{0.5\textwidth}
        \begin{block}{\faCircle \, Zero Vector}
            All entries are zero: $\mathbf{0} = \begin{bmatrix} 0 \\ 0 \\ \vdots \\ 0 \end{bmatrix}$
        \end{block}
        
        \begin{block}{\faSquare \, Ones Vector}
            All entries are one: $\mathbf{1} = \begin{bmatrix} 1 \\ 1 \\ \vdots \\ 1 \end{bmatrix}$
        \end{block}
        
        \begin{block}{\faDotCircle \, Unit Vectors}
            One entry is 1, others are 0:
            \begin{align}
                \mathbf{e}_1 = \begin{bmatrix} 1 \\ 0 \\ 0 \end{bmatrix}, \quad
                \mathbf{e}_2 = \begin{bmatrix} 0 \\ 1 \\ 0 \end{bmatrix}, \quad
                \mathbf{e}_3 = \begin{bmatrix} 0 \\ 0 \\ 1 \end{bmatrix}
            \end{align}
        \end{block}
    \end{column}
    
    \begin{column}{0.45\textwidth}
        \begin{block}{\faEllipsisH \, Sparse Vectors}
            Many entries are zero
            \begin{itemize}
                \item $\text{nnz}(\mathbf{x})$ = number of nonzero entries
                \item Examples: zero vectors, unit vectors
            \end{itemize}
        \end{block}
        
        \begin{block}{\faLayerGroup \, Block Vectors}
            Concatenation of smaller vectors:
            \begin{align}
                \mathbf{a} = \begin{bmatrix}
                    \mathbf{b} \\
                    \mathbf{c} \\
                    \mathbf{d}
                \end{bmatrix}
            \end{align}
        \end{block}
        
        \begin{tikzpicture}[scale=0.6]
            % Unit vectors visualization
            \draw[->] (0,0) -- (2,0) node[below] {$x$};
            \draw[->] (0,0) -- (0,2) node[left] {$y$};
            \draw[->, thick, red] (0,0) -- (1.5,0) node[below] {$\mathbf{e}_1$};
            \draw[->, thick, blue] (0,0) -- (0,1.5) node[left] {$\mathbf{e}_2$};
            \node[below] at (1,-0.3) {\small Standard basis};
        \end{tikzpicture}
    \end{column}
\end{columns}
\end{frame}

\begin{frame}{\faLayerGroup \, Block Vectors and Concatenation}
\framesubtitle{Building larger vectors from smaller components}

\begin{columns}[T]
    \begin{column}{0.6\textwidth}
        \begin{definition}[Stacked Vector]
            Given vectors $\mathbf{b} \in \mathbb{R}^m$, $\mathbf{c} \in \mathbb{R}^n$, and $\mathbf{d} \in \mathbb{R}^p$:
            \begin{align}
                \mathbf{a} = \begin{bmatrix}
                    \mathbf{b} \\
                    \mathbf{c} \\
                    \mathbf{d}
                \end{bmatrix} \in \mathbb{R}^{m+n+p}
            \end{align}
        \end{definition}
        
        \vspace{0.3cm}
        \textbf{Expanded form:}
        \begin{align}
            \mathbf{a} = \begin{bmatrix}
                b_1 \\ b_2 \\ \vdots \\ b_m \\
                c_1 \\ c_2 \\ \vdots \\ c_n \\
                d_1 \\ d_2 \\ \vdots \\ d_p
            \end{bmatrix}
        \end{align}
    \end{column}
    
    \begin{column}{0.35\textwidth}
        \begin{example}[Practical Example]
            \begin{align}
                \mathbf{b} &= \begin{bmatrix} 1 \\ 2 \end{bmatrix} \\[0.3cm]
                \mathbf{c} &= \begin{bmatrix} 3 \\ 4 \\ 5 \end{bmatrix} \\[0.3cm]
                \mathbf{a} &= \begin{bmatrix} \mathbf{b} \\ \mathbf{c} \end{bmatrix} = \begin{bmatrix} 1 \\ 2 \\ 3 \\ 4 \\ 5 \end{bmatrix}
            \end{align}
        \end{example}
        
        \begin{alertblock}{Applications}
            \small
            \begin{itemize}
                \item Feature concatenation in ML
                \item Stacking data samples
                \item Building composite models
            \end{itemize}
        \end{alertblock}
    \end{column}
\end{columns}
\end{frame}

\begin{frame}{\faEye \, Geometric Interpretation}
\framesubtitle{Visualizing vectors in 2D and 3D space}

\begin{columns}[T]
    \begin{column}{0.45\textwidth}
        \textbf{2D Vector Representations:}
        
        \begin{tikzpicture}[scale=0.8]
            % Point representation
            \begin{scope}[xshift=0cm]
                \draw[->] (0,0) -- (3,0) node[below] {$x_1$};
                \draw[->] (0,0) -- (0,3) node[left] {$x_2$};
                \draw[dashed,gray] (2.5,2) -- (2.5,0);
                \draw[dashed,gray] (2.5,2) -- (0,2);
                \fill[blue] (2.5,2) circle (2pt);
                \node[above right] at (2.5,2) {$\mathbf{x}$};
                \node[below] at (1.5,-0.5) {\small Point in plane};
            \end{scope}
        \end{tikzpicture}
        
        \vspace{0.5cm}
        
        \begin{tikzpicture}[scale=0.8]
            % Vector representation
            \begin{scope}[xshift=0cm]
                \draw[->] (0,0) -- (3,0) node[below] {$x_1$};
                \draw[->] (0,0) -- (0,3) node[left] {$x_2$};
                \draw[->,blue,thick] (0,0) -- (2.5,2) node[above right] {$\mathbf{x}$};
                \node[below] at (1.5,-0.5) {\small Displacement vector};
            \end{scope}
        \end{tikzpicture}
    \end{column}
    
    \begin{column}{0.5\textwidth}
        \textbf{Key Geometric Properties:}
        
        \begin{block}{Magnitude (Length)}
            \begin{align}
                \|\mathbf{x}\| = \sqrt{x_1^2 + x_2^2 + \cdots + x_n^2}
            \end{align}
        \end{block}
        
        \begin{block}{Direction}
            The angle a vector makes with coordinate axes
        \end{block}
        
        \begin{example}[Vector $\mathbf{x} = \begin{bmatrix} 3 \\ 4 \end{bmatrix}$]
            \begin{itemize}
                \item Magnitude: $\|\mathbf{x}\| = \sqrt{3^2 + 4^2} = 5$
                \item Direction: $\theta = \arctan(4/3) \approx 53.13°$
            \end{itemize}
        \end{example}
        
        \begin{alertblock}{ML Connection}
            \small Vectors represent feature vectors, embeddings, gradients, etc.
        \end{alertblock}
    \end{column}
\end{columns}
\end{frame}

% End of current presentation content
% Additional sections (Matrices, Decomposition, Tensors) would follow here

\end{document}
