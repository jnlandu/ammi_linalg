\begin{frame}{Getting Started with Python}
    \begin{itemize}
        \item \textbf{Installation Options:}
        \begin{itemize}
            \item \textbf{Anaconda}: Complete data science platform (recommended for beginners)
            \item \textbf{Python.org}: Official Python distribution
            \item \textbf{Google Colab}: Browser-based, no installation required
        \end{itemize}
        
        \item \textbf{Development Environments:}
        \begin{itemize}
            \item \textbf{Jupyter Notebook}: Interactive development and documentation
            \item \textbf{VS Code}: Versatile editor with Python extensions
            \item \textbf{PyCharm}: Full-featured Python IDE
            \item \textbf{Spyder}: Scientific development environment
        \end{itemize}
        
        \item \textbf{First Steps:}
        \begin{itemize}
            \item Open a Python interpreter or Jupyter notebook
            \item Try the classic first program: \pycode{print("Hello, World!")}
        \end{itemize}
    \end{itemize}
\end{frame}

\begin{frame}[fragile]{Your First Python Program}
    \begin{codeblock}[Hello World Example]
        \begin{lstlisting}
# This is a comment - Python ignores this line
print("Hello, World!")
print("Welcome to Python for AI/ML!")

# Variables and basic operations
name = "AMMI Student"
year = 2025
print(f"Hello {name}, welcome to {year}!")

# Simple calculation
result = 10 + 5 * 2
print(f"The result is: {result}")
        \end{lstlisting}
    \end{codeblock}
    
    \textbf{Output:}
    \begin{block}{}
        \footnotesize
        \texttt{Hello, World!} \\
        \texttt{Welcome to Python for AI/ML!} \\
        \texttt{Hello AMMI Student, welcome to 2025!} \\
        \texttt{The result is: 20}
    \end{block}
\end{frame}

% \begin{frame}{Python Syntax Fundamentals}
%     \begin{itemize}
%         \item \textbf{Indentation Matters:} Python uses indentation to define code blocks
%         \begin{itemize}
%             \item Use 4 spaces (recommended) or tabs consistently
%             \item No curly braces \{\} like other languages
%         \end{itemize}
        
%         \item \textbf{Case Sensitive:} \pycode{Variable} and \pycode{variable} are different
        
%         \item \textbf{Comments:}
%         \begin{itemize}
%             \item Single line: \pycode{# This is a comment}
%             \item Multi-line: \pycode{"""This is a multi-line comment"""}
%         \end{itemize}
        
%         \item \textbf{Line Continuation:}
%         \begin{itemize}
%             \item Use backslash \pycode{\\} or parentheses for long lines
%             \item Python statements typically end with a newline
%         \end{itemize}
%     \end{itemize}
% \end{frame}

\begin{frame}[fragile]{Interactive Python - REPL}
    \begin{itemize}
        \item \textbf{REPL}: Read-Eval-Print-Loop for interactive programming
        \item Great for testing ideas and learning
    \end{itemize}
    
    \begin{codeblock}[Python Interactive Session]
        \begin{lstlisting}
>>> 2 + 3
5
>>> name = "Python"
>>> print(f"I love {name}!")
I love Python!
>>> import math
>>> math.sqrt(16)
4.0
>>> help(print)  # Get help on any function
        \end{lstlisting}
    \end{codeblock}
    
    \begin{exercise}
        Open a Python interpreter and try:
        \begin{enumerate}
            \item Calculate \pycode{2 ** 10} (2 to the power of 10)
            \item Create a variable with your name and print a greeting
            \item Use \pycode{help(len)} to learn about the len function
        \end{enumerate}
    \end{exercise}
\end{frame}
