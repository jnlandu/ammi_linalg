\begin{frame}{Python Best Practices for Data Science}
    \begin{itemize}
        \item \textbf{Code Style (PEP 8):}
        \begin{itemize}
            \item Use 4 spaces for indentation
            \item Keep lines under 79 characters
            \item Use descriptive variable names: \pycode{student_grades} not \pycode{sg}
            \item Follow naming conventions: \pycode{snake_case} for variables and functions
        \end{itemize}
        
        \item \textbf{Documentation:}
        \begin{itemize}
            \item Write docstrings for functions and classes
            \item Use comments for complex logic
            \item Keep README files for projects
        \end{itemize}
        
        \item \textbf{Code Organization:}
        \begin{itemize}
            \item Use functions to avoid code repetition
            \item Organize related code into modules
            \item Separate configuration from code
        \end{itemize}
    \end{itemize}
\end{frame}

\begin{frame}[fragile]{Writing Clean, Readable Code}
    \begin{codeblock}[Best Practices Example]
        \begin{lstlisting}
# Good: Clear, well-documented function
def calculate_student_average(grades):
    """
    Calculate the average grade for a student.
    
    Args:
        grades (list): List of numeric grades
        
    Returns:
        float: Average grade, or 0 if no grades provided
    """
    if not grades:
        return 0
    
    total = sum(grades)
    count = len(grades)
    average = total / count
    
    return round(average, 2)

# Good: Descriptive variable names
student_names = ["Alice", "Bob", "Charlie"]
final_grades = [85, 92, 78]

# Good: Using list comprehension appropriately
passing_students = [
    name for name, grade in zip(student_names, final_grades) 
    if grade >= 70
]

print(f"Passing students: {passing_students}")
        \end{lstlisting}
    \end{codeblock}
\end{frame}
