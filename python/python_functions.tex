\begin{frame}[fragile]{Defining Functions}
    \begin{codeblock}[Function Basics]
        \begin{lstlisting}
# Basic function definition
def greet(name):
    """This function greets someone."""
    return f"Hello, {name}!"

# Calling the function
message = greet("Alice")
print(message)  # Hello, Alice!

# Function with multiple parameters
def calculate_area(length, width):
    """Calculate rectangle area."""
    area = length * width
    return area

result = calculate_area(5, 3)
print(f"Area: {result}")

# Function with default parameters
def power(base, exponent=2):
    """Calculate base raised to exponent (default: 2)."""
    return base ** exponent

print(power(4))      # 16 (4^2)
print(power(4, 3))   # 64 (4^3)

# Function with variable arguments
def sum_all(*args):
    """Sum all provided arguments."""
    return sum(args)

print(sum_all(1, 2, 3, 4, 5))  # 15
        \end{lstlisting}
    \end{codeblock}
\end{frame}

\begin{frame}{Modules and Packages}
    \begin{itemize}
        \item \textbf{Importing Modules:}
        \begin{itemize}
            \item \pycode{import math} - Import entire module
            \item \pycode{from math import sqrt} - Import specific function
            \item \pycode{import numpy as np} - Import with alias
        \end{itemize}
        
        \item \textbf{Standard Library Modules:}
        \begin{itemize}
            \item \textbf{math}: Mathematical functions
            \item \textbf{random}: Random number generation
            \item \textbf{datetime}: Date and time handling
            \item \textbf{os}: Operating system interface
            \item \textbf{json}: JSON data handling
        \end{itemize}
        
        \item \textbf{Creating Your Own Modules:}
        \begin{itemize}
            \item Save functions in a \texttt{.py} file
            \item Import using the filename (without \texttt{.py})
            \item Use \pycode{if __name__ == "__main__":} for executable scripts
        \end{itemize}
    \end{itemize}
\end{frame}
