\begin{frame}{Python Data Types Overview}
    \begin{columns}
        \begin{column}{0.5\textwidth}
            \textbf{Basic Types:}
            \begin{itemize}
                \item \textbf{int}: Integers (1, 42, -5)
                \item \textbf{float}: Decimal numbers (3.14, -2.5)
                \item \textbf{str}: Text strings ("Hello", 'Python')
                \item \textbf{bool}: True or False
            \end{itemize}
            
            \textbf{Collection Types:}
            \begin{itemize}
                \item \textbf{list}: Ordered, mutable [1, 2, 3]
                \item \textbf{tuple}: Ordered, immutable (1, 2, 3)
                \item \textbf{dict}: Key-value pairs \{"key": "value"\}
                \item \textbf{set}: Unique elements \{1, 2, 3\}
            \end{itemize}
        \end{column}
        
        \begin{column}{0.5\textwidth}
            \begin{figure}
                \begin{center}
                \begin{tikzpicture}[scale=0.7]
                    % Basic types
                    \node[draw, rounded corners, fill=pythonblue!20] (basic) at (0,2) {Basic Types};
                    \node[below=0.3cm of basic, font=\footnotesize] {int, float, str, bool};
                    
                    % Collections
                    \node[draw, rounded corners, fill=pythongreen!20] (collections) at (0,0) {Collections};
                    \node[below=0.3cm of collections, font=\footnotesize] {list, tuple, dict, set};
                    
                    % Advanced
                    \node[draw, rounded corners, fill=pythonyellow!20] (advanced) at (0,-2) {Advanced};
                    \node[below=0.3cm of advanced, font=\footnotesize] {functions, classes, modules};
                \end{tikzpicture}
                \end{center}
                \caption{Python type hierarchy}
            \end{figure}
        \end{column}
    \end{columns}
\end{frame}

\begin{frame}[fragile]{Numbers and Basic Operations}
    \begin{codeblock}[Numeric Types and Operations]
        \begin{lstlisting}
# Integer operations
a = 10
b = 3
print(f"Addition: {a + b}")        # 13
print(f"Subtraction: {a - b}")     # 7
print(f"Multiplication: {a * b}")  # 30
print(f"Division: {a / b}")        # 3.333...
print(f"Floor division: {a // b}") # 3
print(f"Modulus: {a % b}")         # 1
print(f"Exponentiation: {a ** b}") # 1000

# Float operations
pi = 3.14159
radius = 5.0
area = pi * radius ** 2
print(f"Circle area: {area:.2f}")  # 78.54

# Type checking
print(type(a))        # <class 'int'>
print(type(pi))       # <class 'float'>
print(isinstance(a, int))  # True
        \end{lstlisting}
    \end{codeblock}
\end{frame}

\begin{frame}[fragile]{Strings and Text Processing}
    \begin{codeblock}[String Operations]
        \begin{lstlisting}
# String creation
name = "Alice"
greeting = 'Hello'
message = """This is a
multi-line string"""

# String formatting
age = 25
formatted = f"My name is {name} and I am {age} years old"
print(formatted)

# String methods
text = "  Python Programming  "
print(text.strip())           # Remove whitespace
print(text.upper())           # PYTHON PROGRAMMING
print(text.lower())           # python programming
print(text.replace("Python", "Java"))  # Java Programming

# String slicing
word = "Machine Learning"
print(word[0:7])      # "Machine"
print(word[8:])       # "Learning"
print(word[::-1])     # Reverse string
        \end{lstlisting}
    \end{codeblock}
\end{frame}

\begin{frame}[fragile]{Variables and Assignment}
    \begin{codeblock}[Variable Assignment and Naming]
        \begin{lstlisting}
# Variable assignment
x = 5
y = 10
z = x + y

# Multiple assignment
a, b, c = 1, 2, 3
first_name, last_name = "John", "Doe"

# Variable naming conventions (PEP 8)
student_name = "Alice"      # Good: snake_case
CONSTANT_VALUE = 3.14159    # Good: UPPER_CASE for constants
class_size = 30             # Good: descriptive names

# Avoid these naming styles
# 1student = "Bob"          # Error: starts with number
# student-name = "Charlie"  # Error: hyphens not allowed
# class = "ML"              # Error: reserved keyword

# Dynamic typing
var = 42        # var is an integer
print(type(var))
var = "Hello"   # now var is a string
print(type(var))
var = [1, 2, 3] # now var is a list
print(type(var))
        \end{lstlisting}
    \end{codeblock}
\end{frame}

\begin{frame}[fragile]{Boolean Logic and Comparisons}
    \begin{codeblock}[Boolean Operations]
        \begin{lstlisting}
# Boolean values
is_student = True
is_graduated = False

# Comparison operators
x, y = 10, 20
print(x == y)    # False (equal)
print(x != y)    # True (not equal)
print(x < y)     # True (less than)
print(x <= y)    # True (less than or equal)
print(x > y)     # False (greater than)
print(x >= y)    # False (greater than or equal)

# Logical operators
age = 22
has_license = True
can_drive = age >= 18 and has_license
print(can_drive)  # True

# Truthiness in Python
print(bool(0))        # False
print(bool(1))        # True
print(bool(""))       # False (empty string)
print(bool("Hello"))  # True
print(bool([]))       # False (empty list)
print(bool([1, 2]))   # True
        \end{lstlisting}
    \end{codeblock}
\end{frame}
