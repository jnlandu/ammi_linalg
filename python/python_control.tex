\begin{frame}[fragile]{Conditional Statements - if/elif/else}
    \begin{codeblock}[Conditional Logic]
        \begin{lstlisting}
# Basic if statement
score = 85

if score >= 90:
    grade = "A"
    print("Excellent work!")
elif score >= 80:
    grade = "B"
    print("Good job!")
elif score >= 70:
    grade = "C"
    print("Satisfactory")
else:
    grade = "F"
    print("Need improvement")

print(f"Your grade is: {grade}")

# Ternary operator (one-line if)
status = "Pass" if score >= 60 else "Fail"
print(f"Status: {status}")

# Multiple conditions
age = 20
has_id = True
can_enter = age >= 18 and has_id
if can_enter:
    print("Welcome!")
else:
    print("Access denied")
        \end{lstlisting}
    \end{codeblock}
\end{frame}

\begin{frame}[fragile]{Loops - for and while}
    \begin{codeblock}[Iteration Structures]
        \begin{lstlisting}
# For loop with range
print("Counting from 0 to 4:")
for i in range(5):
    print(i)

# For loop with list
fruits = ["apple", "banana", "orange"]
print("\nFruits in our basket:")
for fruit in fruits:
    print(f"- {fruit}")

# For loop with enumerate (index + value)
print("\nIndexed fruits:")
for index, fruit in enumerate(fruits):
    print(f"{index}: {fruit}")

# While loop
count = 0
print("\nCountdown:")
while count < 3:
    print(f"Count: {count}")
    count += 1  # Same as count = count + 1

print("Done!")
        \end{lstlisting}
    \end{codeblock}
\end{frame}

\begin{frame}[fragile]{Loop Control and List Comprehensions}
    \begin{codeblock}[Advanced Loop Techniques]
        \begin{lstlisting}
# Loop control statements
numbers = [1, 2, 3, 4, 5, 6, 7, 8, 9, 10]

print("Even numbers:")
for num in numbers:
    if num % 2 == 0:
        print(num)
    
print("\nNumbers less than 6:")
for num in numbers:
    if num >= 6:
        break  # Exit the loop
    print(num)

print("\nOdd numbers (using continue):")
for num in numbers:
    if num % 2 == 0:
        continue  # Skip to next iteration
    print(num)

# List comprehensions (Pythonic way)
squares = [x**2 for x in range(1, 6)]
print(f"Squares: {squares}")

even_squares = [x**2 for x in range(1, 11) if x % 2 == 0]
print(f"Even squares: {even_squares}")
        \end{lstlisting}
    \end{codeblock}
\end{frame}

\begin{frame}[fragile]{Nested Loops and Patterns}
    \begin{codeblock}[Nested Loop Examples]
        \begin{lstlisting}
# Multiplication table
print("Multiplication Table (3x3):")
for i in range(1, 4):
    for j in range(1, 4):
        product = i * j
        print(f"{i}x{j}={product:2d}", end="  ")
    print()  # New line after each row

# Pattern printing
print("\nStar pattern:")
for i in range(1, 6):
    print("*" * i)

# Matrix creation using nested loops
matrix = []
for i in range(3):
    row = []
    for j in range(3):
        row.append(i * 3 + j + 1)
    matrix.append(row)

print("\nMatrix:")
for row in matrix:
    print(row)

# Same matrix using list comprehension
matrix_comp = [[i*3 + j + 1 for j in range(3)] for i in range(3)]
print(f"\nUsing comprehension: {matrix_comp}")
        \end{lstlisting}
    \end{codeblock}
\end{frame}

\begin{frame}{Control Flow Best Practices}
    \begin{itemize}
        \item \textbf{Readable Conditions:}
        \begin{itemize}
            \item Use descriptive variable names: \pycode{is_valid} instead of \pycode{flag}
            \item Combine related conditions: \pycode{if 18 <= age <= 65:}
        \end{itemize}
        
        \item \textbf{Loop Efficiency:}
        \begin{itemize}
            \item Prefer \pycode{for} loops over \pycode{while} when possible
            \item Use \pycode{enumerate()} instead of \pycode{range(len())}
            \item Consider list comprehensions for simple transformations
        \end{itemize}
        
        \item \textbf{Avoid Common Pitfalls:}
        \begin{itemize}
            \item Don't modify a list while iterating over it
            \item Use \pycode{break} and \pycode{continue} judiciously
            \item Be careful with infinite loops in \pycode{while} statements
        \end{itemize}
        
        \item \textbf{Code Style:}
        \begin{itemize}
            \item Use 4 spaces for indentation (PEP 8)
            \item Keep line length under 79 characters
            \item Add comments for complex logic
        \end{itemize}
    \end{itemize}
\end{frame}
