\begin{frame}{Data Preprocessing Essentials}
    \begin{itemize}
        \item \textbf{Why Preprocessing?} Raw data is often messy and inconsistent
        \item \textbf{Common Tasks:}
        \begin{itemize}
            \item Handling missing values
            \item Scaling and normalization
            \item Encoding categorical variables
            \item Feature selection and extraction
        \end{itemize>
        \item \textbf{Rule of Thumb:} 80% of data science is data cleaning and preprocessing
    \end{itemize>
\end{frame>

\begin{frame}[fragile]{Data Cleaning Techniques}
    \begin{codeblock}[Preprocessing Examples]
        \begin{lstlisting}
import pandas as pd
from sklearn.preprocessing import StandardScaler, LabelEncoder
from sklearn.impute import SimpleImputer

# Sample messy data
data = {
    'age': [25, 30, None, 35, 28],
    'salary': [50000, 60000, 55000, None, 52000],
    'department': ['IT', 'HR', 'IT', 'Finance', 'HR']
}
df = pd.DataFrame(data)

# Handle missing values
imputer = SimpleImputer(strategy='mean')
df[['age', 'salary']] = imputer.fit_transform(df[['age', 'salary']])

# Encode categorical variables
label_encoder = LabelEncoder()
df['department_encoded'] = label_encoder.fit_transform(df['department'])

# Scale numerical features
scaler = StandardScaler()
df[['age_scaled', 'salary_scaled']] = scaler.fit_transform(
    df[['age', 'salary']]
)

print(df)
        \end{lstlisting>
    \end{codeblock>
\end{frame>
