\subsection{Definitions and Notations}

\begin{frame}{Vectors - Definition and Notation}
    \begin{itemize}
        \item A \textit{vector} is an ordered list of numbers.
        \item Written in \textbf{row representation} as:
        \begin{align}
            \left[ x_1, x_2, \cdots, x_n, \ldots \right] \text{ or } \left( x_1, x_2, \cdots, x_n, \ldots \right) \label{eq:row}
        \end{align} 
        \item Or in \textbf{column representation} as:
        \begin{align}
            \begin{bmatrix}
                x_1 \\ x_2 \\ \vdots \\ x_n \\ \vdots
            \end{bmatrix} \text{ or }
            \begin{pmatrix}
                x_1 \\ x_2 \\ \vdots \\ x_n \\ \vdots
            \end{pmatrix} \label{eq:column}
        \end{align}
    \end{itemize}
\end{frame}

\begin{frame}{Vector Components and Dimensions}
    \begin{itemize}
        \item Quantities $x_1, x_2, \ldots, x_n$ are called \textbf{components} (elements, coefficients, entries).
        \item The number of elements is the \textbf{dimension} (or length) of the vector.
        \item Examples of finite vectors:
        \begin{align}
            \begin{bmatrix} 1 \\ 2 \\ -1 \end{bmatrix} \text{ and }
            \begin{bmatrix} -2 \\ 1.1 \\ 0 \\ 0 \end{bmatrix} \label{eq:examples}
        \end{align}
        are vectors of size 3 and 4, respectively.
    \end{itemize}
\end{frame}

\begin{frame}{Vector Notation Conventions}
    \begin{itemize}
        \item \textbf{Scalars:} Numbers are vectors of dimension 0.
        \item \textbf{Notation:} Lowercase letters denote vectors: $a, x, p, r$ or $\vec{x}$.
        \item \textbf{Indexing:} The $i$-th element of vector $x$ is denoted $x_i$.
        \item \textbf{Equality:} Vectors $x$ and $y$ are equal ($x = y$) if and only if $x_i = y_i$ for all $i$.
        \item \textbf{Dimension notation:} 
            \begin{itemize}
                \item $n$-vector: vector of dimension $n$
                \item $(n,1)$-column vector or $(1,n)$-row vector
            \end{itemize}
    \end{itemize}
\end{frame}

\begin{frame}{Block Vectors and Concatenation}
    \begin{itemize}
        \item \textbf{Stacked vector:} For vectors $b, c, d$ of sizes $m, n, p$:
        \begin{align}
            a = \begin{bmatrix} b \\ c \\ d \end{bmatrix}
        \end{align}
        \item Also called \textbf{block vector} with block entries $b, c, d$.
        \item Result: $a$ is an $(m+n+p)$-vector.
        \item \textbf{Expanded form:}
        \begin{align*}
            a = (b_1, b_2, \ldots, b_m, c_1, c_2, \ldots, c_n, d_1, d_2, \ldots, d_p)
        \end{align*}
    \end{itemize}
\end{frame}

\begin{frame}{Special Vectors}
    \begin{itemize}
        \item \textbf{Zero vector:} All entries are zero, denoted $\mathbf{0}$.
        \item \textbf{Ones vector:} All entries are 1, denoted $\mathbf{1}$.
        \item \textbf{Unit vectors:} One entry is 1, all others are 0, denoted $e_i$.
        \item \textbf{Example} - Unit vectors in dimension 3:
        \begin{align}
            e_1 = \begin{bmatrix} 1 \\ 0 \\ 0 \end{bmatrix}, \quad
            e_2 = \begin{bmatrix} 0 \\ 1 \\ 0 \end{bmatrix}, \quad
            e_3 = \begin{bmatrix} 0 \\ 0 \\ 1 \end{bmatrix} \label{eq:unit_vectors}
        \end{align}
        \item \textbf{Sparse vectors:} Many entries are zero. We denote $\text{nnz}(x)$ as the number of nonzero entries.
    \end{itemize}
\end{frame}

\begin{frame}{Geometric Representation}
    \begin{itemize}
        \item A 2D vector $(x_1, x_2)$ represents a location or displacement in a plane:
    \end{itemize}
    
    \begin{figure}[h]
        \centering
        \begin{tikzpicture}
            % First diagram: Point representation
            \draw[->] (0,0) -- (3,0) node[below] {$x_1$};
            \draw[->] (0,0) -- (0,3) node[left] {$x_2$};
            \draw[dashed,gray] (2.5,2.5) -- (2.5,0);
            \draw[dashed,gray] (2.5,2.5) -- (0,2.5);
            \fill[blue] (2.5,2.5) circle (2pt);
            \node[right] at (2.5,2.5) {$x$};
        \end{tikzpicture}
        \hspace{2cm}
        \begin{tikzpicture}
            % Second diagram: Vector representation
            \draw[->] (0,0) -- (3,0) node[below] {$x_1$};
            \draw[->] (0,0) -- (0,3) node[left] {$x_2$};
            \draw[->,blue,thick] (0,0) -- (2.5,2.5) node[right] {$x$};
        \end{tikzpicture}
        \caption{Point vs. Vector representation in 2D}
    \end{figure}
\end{frame}

\begin{frame}{Real-World Vector Examples}
    \begin{itemize}
        \item \textbf{Economics:} Quantities of $n$ different commodities
        \item \textbf{Education:} Student grades per subject/year
        \item \textbf{Finance:} Cash flow where $x_i$ is payment in period $i$
        \item \textbf{Audio:} Acoustic pressure at sample time $i$
        \item \textbf{NLP:} Word count where $x_i$ is frequency of word $i$ in a document
    \end{itemize}
\end{frame}

\begin{frame}{Text Processing Example}
    \begin{quote}
        \textcolor{red}{
        "I have a \textcolor{black}{dream} that one day \textcolor{black}{every} valley \textcolor{cyan}{shall} \textcolor{black}{be} exalted, \textcolor{black}{every} hill \textcolor{black}{and} mountain \textcolor{cyan}{shall}  \textcolor{black}{be} \textcolor{black}{made} low, 
        the rough places will \textcolor{black}{be} \textcolor{black}{made} plain, \textcolor{black}{and} the crooked places will \textcolor{black}{be} \textcolor{black}{made} straight; 
        and the glory of the Lord \textcolor{cyan}{shall}  \textcolor{black}{be} revealed, and all flesh \textcolor{cyan}{shall}  see it together."
        }
    \end{quote}
    \noindent — Martin Luther King Jr., \textcolor{blue}{\textbf{"I have a dream"}} (excerpt)
\end{frame}

\begin{frame}{Word Count Vector Example}
    \begin{itemize}
        \item Dictionary and corresponding word count vector:
        \begin{align*}
            \begin{array}{r}
                \text{dream} \\
                \text{every} \\
                \text{be} \\
                \text{shall} \\
                \text{made}
            \end{array} \quad \rightarrow \quad 
            \begin{bmatrix}
                1 \\ 2 \\ 3 \\ 4 \\ 3
            \end{bmatrix}
        \end{align*}
        \item In practice, dictionaries contain thousands of words.
    \end{itemize}
\end{frame}