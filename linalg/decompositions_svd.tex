
\begin{frame}{SVD Theorem}
\begin{itemize}
    \item Let \( A \in \mathbb{R}^{m \times n} \) be a matrix of rank \( r \leq \min(m, n) \). The Singular Value Decomposition (SVD) of \( A \) expresses it as:

\[
A = U \Sigma V^T
\]

where:
\begin{itemize}
    \item \( U \in \mathbb{R}^{m \times m} \) is an orthogonal matrix with columns \( u_i \) (left singular vectors)
    \item \( V \in \mathbb{R}^{n \times n} \) is an orthogonal matrix with columns \( v_j \) (right singular vectors)
    \item \( \Sigma \in \mathbb{R}^{m \times n} \) has diagonal entries \( \sigma_i \geq 0 \) and zeros elsewhere
\end{itemize}
\end{itemize}
\end{frame}

\begin{frame}{Properties of SVD}
\begin{itemize}
    \item The diagonal entries $\sigma_1 \geq \sigma_2 \geq \ldots \geq \sigma_r > 0$ are called \textbf{singular values} of $A$.
    \item The columns $u_i$ are \textbf{left singular vectors}, and columns $v_j$ are \textbf{right singular vectors}.
    \item The singular values are conventionally ordered in decreasing order.
    \item The matrix $\Sigma$ is uniquely determined. For $m > n$:
    \begin{align}
        \Sigma = \begin{bmatrix}
            \sigma_1 & 0 & \cdots & 0 \\
            0 & \sigma_2 & \cdots & 0 \\
            \vdots & \vdots & \ddots & \vdots \\
            0 & 0 & \cdots & \sigma_n \\
            0 & 0 & \cdots & 0 \\
            \vdots & \vdots & \ddots & \vdots \\
            0 & 0 & \cdots & 0
        \end{bmatrix}
    \end{align}
\end{itemize}
\end{frame}

\begin{frame}{}
\begin{itemize}
    \item For $m < n$, $\Sigma$ has the structure:
    \begin{align}
        \Sigma = \begin{bmatrix}
            \sigma_1 & 0 & \cdots & 0 & 0 & \cdots & 0 \\
            0 & \sigma_2 & \cdots & 0 & 0 & \cdots & 0\\
            \vdots & \vdots & \ddots & \vdots & \vdots & \ddots & \vdots \\
            0 & 0 & \cdots & \sigma_m & 0 & \cdots & 0
        \end{bmatrix}
    \end{align}
    \item \textbf{Key fact:} The SVD exists for \emph{any} matrix $A \in \mathbb{R}^{m \times n}$, regardless of rank or shape.
\end{itemize}
\end{frame}

\begin{frame}{Computing the SVD}
\begin{itemize}
    \item The SVD can be constructed using the eigendecomposition of $A^TA$:
    \begin{align}
        A^TA = Q\Lambda Q^T = Q \begin{bmatrix}
            \lambda_1 & \cdots & 0\\
            \vdots & \ddots & \vdots\\
            0 & \cdots & \lambda_n
        \end{bmatrix} Q^T
    \end{align}
    where $Q$ contains orthonormal eigenvectors and $\lambda_i \geq 0$ are eigenvalues of $A^TA$.
    \item Starting from the SVD form $A = U\Sigma V^T$, we compute:
    \begin{align}
        A^TA = (U\Sigma V^T)^T (U\Sigma V^T) = V\Sigma^T U^T U \Sigma V^T
    \end{align}
\end{itemize}
\end{frame}

\begin{frame}{}
\begin{itemize}
    \item Since $U$ is orthogonal ($U^TU = I$), we get:
    \begin{align}
        A^TA = V\Sigma^T \Sigma V^T = V \begin{bmatrix}
            \sigma_1^2 & & \\
            & \ddots & \\
            & & \sigma_n^2
        \end{bmatrix} V^T
    \end{align}
    \item This shows that the singular values are $\sigma_i = \sqrt{\lambda_i}$, where $\lambda_i$ are eigenvalues of $A^TA$.
\end{itemize}
\end{frame}