\begin{frame}{Determinants}
\begin{itemize}
    \item For any square matrix $A \in \mathbb{R}^{n\times n}$, the determinant $\det(A)$ (also written as $|A|$) is a scalar value given by
    \begin{align}
        \det (A) = \left\vert\begin{matrix}
            a_{11} & a_{12} & \cdots & a_{1n}\\
            a_{21} & a_{22} & \cdots & a_{2n} \\
            \vdots & \vdots & \ddots & \vdots \\
            a_{n1} & a_{n2} &\cdots & a_{nn}
        \end{matrix}\right\vert 
    \end{align}
    \item Determinants are only defined for square matrices.
\end{itemize}
\end{frame}


\begin{frame}{}
\begin{itemize}
    \item For any $n \times n$ matrix $A$, we can compute $\det A$ using cofactor expansion:
    \begin{align}
        \det A = \sum_{i=1}^{n} (-1)^{i+j}a_{ij} \det A_{ij} \; \text{ or }   \det A = \sum_{j=1}^{n} (-1)^{i+j}a_{ij} \det A_{ij} \label{eq3,16}
    \end{align}
    where $a_{ij}$ is the element at row $i$ and column $j$, and $A_{ij}$ is the $(i,j)$-minor of matrix $A$.
    \item For a $2 \times 2$ matrix $A$, we have:
    \begin{align}
        \det A  = \left\vert\begin{matrix}
            a_{11} & a_{12} \\
            a_{21} & a_{22}  \\
        \end{matrix}\right\vert = a_{11}a_{22} - a_{12}a_{21}
    \end{align}
\end{itemize}
\end{frame}
\begin{frame}{}
\begin{itemize}
    \item For a $3 \times 3$ matrix $A$, applying Eq. \ref{eq3,16} gives:
    \begin{align*}
        \det A = \left\vert\begin{matrix}
            a_{11} & a_{12} &  a_{13}\\
            a_{21} & a_{22} & a_{23} \\
            a_{31} & a_{32} &a_{33}
        \end{matrix}\right\vert  &= a_{11} \left\vert\begin{matrix}
            a_{22} & a_{23} \\
            a_{32} & a_{33}  \\
        \end{matrix}\right\vert  -\\
        & a_{12}\left\vert\begin{matrix}
            a_{21} & a_{23} \\
            a_{31} & a_{33}  \\
        \end{matrix}\right\vert + a_{13} \left\vert\begin{matrix}
            a_{21} & a_{22} \\
            a_{31} & a_{32}  \\
        \end{matrix}\right\vert
    \end{align*}
    \item The determinant serves as a test for matrix invertibility.
\end{itemize}
\end{frame}
\begin{frame}{Determinant and Matrix Invertibility}
\begin{itemize}
    \item A square matrix $A\in \mathbb{R}^{n\times n}$ is invertible if and only if $\det(A) \neq 0$.
    \item Proof: Left as an exercise.
    \item Example: Consider the matrix 
    \begin{align}
       A=\begin{bmatrix} 1 & 2 & 3 \\ 0 & 1 & 4 \\ 5 & 6 & 0 \end{bmatrix}
    \end{align}
    \item We calculate: $\det A = 1(1 \cdot 0 - 4 \cdot 6) - 2(0 \cdot 0 - 4 \cdot 5) + 3(0 \cdot 6 - 1 \cdot 5) = -24 + 40 - 15 = 1$.
    \item Since $\det A = 1 \neq 0$, the matrix $A$ is invertible.
\end{itemize}
\end{frame}


\begin{frame}{Properties}
\begin{itemize}
    \item Matrix determinants satisfy the following properties:
    \begin{enumerate}
        \item $\det(AB) = \det(A) \times \det (B)$
        \item  $\det(A) = \det(A^T)$
        \item If $A$ is invertible, then  $\det(A^{-1}) = \frac{1}{\det (A)}$.
        \item $\det (\lambda A ) = \lambda^n \det (A)$
        \item Exchanging two rows or columns negates $\det(A)$.
    \end{enumerate}
\end{itemize}
\end{frame}

\begin{frame}{Trace}
\begin{itemize}
    \item The trace of $A\in \mathbb{R}^{n\times n}$ is the scalar defined as 
    \begin{align}
        \text{tr}(A): = \sum_{i=1}^n a_{ii}
    \end{align}
    \item In other words, the trace equals the sum of all diagonal entries of $A$.
    \item It has the following properties: (Exercise)
    \begin{enumerate}
        \item $\text{tr}(A+B) = \text{tr}(A) + \text{tr}(B)$ for $A, B\in \mathbb{R}^{n\times n}$
        \item $\text{tr}(\lambda A ) = \lambda \text{tr}(A)$
        \item $\text{tr}(I_n) = n$
        \item $\text{tr}(AB) = \text{tr}(BA)$ for all $A\in \mathbb{R}^{n\times k }$, $B\in \mathbb{R}^{k\times n}$.
        \end{enumerate}
     \item The trace remains unchanged under cyclic permutations:
    \begin{align}
        \text{tr}(AKL) = \text{tr}(KLA)
    \end{align}
    for matrices $A\in \mathbb{R}^{a\times k}, \; K \in \mathbb{R}^{k\times l}, \; L \in \mathbb{R}^{l\times a} $
\end{itemize}
\end{frame}
